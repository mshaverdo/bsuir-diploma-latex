\sectioncentered*{Введение}
\addcontentsline{toc}{section}{Введение}
\label{sec:intro}

Из глубины веков доходят до нас легенды о колдунах и магах, но, согласно третьему закону А. Кларка, «любая достаточно развитая технология неотличима от магии». Таким образом, в настоящее время магию современности создают автоматизированные системы управления, яркими представителями которых являются нечеткие контроллеры.

Нечеткие контроллеры обеспечивают эффективное управление процессами. Они предлагают использовать понятное эксперту представление, которое облегчает передачу знаний между системой и пользователями.

К нечетким контроллерам при проектировании предъявляется целый ряд технических и экономических требований, главными из которых являются:

\begin{itemize}
  \itemобеспечить заданное качество управления;
  \itemобеспечить управление в реальном времени; 
  \itemминимизировать затраты применяемой аппаратной платформы.
\end{itemize}

Первое требование может характеризоваться такими свойствами, как плавность изменения регулируемого параметра, устойчивость от шумов и флуктуаций, адаптивность, обучаемость и др., в зависимости от конкретной решаемой задачи.

Любое системной управление предполагает выделение ресурсов и времени на выполнение задач таким образом, чтобы выполнялись определенные требования к производительности: для заданного объекта управления (исполнительного устройства) необходимо выполнить алгоритм управления установленное количество раз в единицу времени. 

Построение системы управления, в которой выдерживаются все вышеперечисленные требования является нетривиальной задачей и неразрывно связано с непрерывной оценкой всех трех параметров в процессе разработки системы. Использование реального аппаратного обеспечения для контроля качества управления и измерения требуемых ресурсов является трудоемкой, дорогой, а подчас и невозможной задачей.

В связи с этим, целью данного дипломного проекта является разработка и реализация системы, позволяющей отлаживать алгоритмы нечеткого управления в симулируемой среде, что позволит существенно упростить и удешевить процесс разработки и отладки алгоритмов управления по сравнению с натурными испытаниями.	

В соответствии с поставленной целью были определены следующие задачи:

\begin{itemize}
  \itemвыбор средств разработки реализации системы;
  \itemразработка и реализация симуляции физической среды;
  \itemреализация инструментов выгрузки телеметрии симулируемой среды;
  \itemразработка средств визуализации данных телеметрии;
  \itemразработка алгоритма настройки нечеткого контроллера.
\end{itemize}