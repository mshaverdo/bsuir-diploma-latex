\sectioncentered*{Реферат}
\thispagestyle{empty}
%%
%% ВНИМАНИЕ: этот реферат не соответствует СТП-01 2013
%% пример оформления реферата смотрите здесь: http://www.bsuir.by/m/12_100229_1_91132.docx 
%%

Дипломный проект представлен следующим образом. Электронные
носители: 1 компакт-диск. Чертежный материал: 6 листов формата А1.
Пояснительная записка: \pageref*{LastPage}~страниц, \totfig{}~рисунков, 10 литературных
источников, 4 приложения.

Ключевые слова: нечеткий контроллер, алгоритм Мамдани, робот,
лингвистическая переменная, терм, моделирование, Python, CLI.

Предметной областью являются нечеткие контроллеры, алгоритмы их
моделирования и настройки. Объектом разработки является программное
средство, позволяющее настраивать и обучать нечеткие контроллеры
мобильных роботов.

Целью разработки является создание программного средства, которое
позволит проводить эксперименты с нечеткими контроллерами мобильных
роботов в моделируемой среде, позволяющее существенно снизить
трудоемкость и стоимость исследования и отладки нечетких алгоритмов.

Для разработки использовался язык Python, пакет для визуализации
научной графики matplotlib, пакетный менеджер pip.

В результате работы над проектом было разработано программное
средство, позволяющее отлаживать алгоритмы нечеткого управления
мобильными роботами. Оно облегчает и автоматизирует настройку нечетких
алгоритмов управления мобильными роботами.

Областью практического применения данного программного средства
является его использование в отладке мобильных платформ предприятиями-
изготовителями, проведение теоретических исследований нечетких
алгоритмов в научно-исследовательских институтах, использование в
преподавательской деятельности.

Разработанный проект является экономически эффективным как для
разработчика, так и для конечного пользователя: он позволяет существенно
сократить материальные и трудозатраты на отладку нечетких алгоритмов.

Дипломный проект является завершенным, поставленная задача
решена в полной мере, присутствует возможность дальнейшего развития
приложения и расширения его функционала. 
\clearpage
